\documentclass[a4paper,10pt]{article}
\usepackage[utf8]{inputenc}

%opening
\title{Black Scholes Calculation}
\author{Cannon Johns}

\begin{document}

\maketitle

\begin{abstract}

The purpose of this document is to act as an exploratory look into the Black Scholes model, a mathematical model which seeks to explain the pricing behavior of financial derivatives. This model is most often used for pricing options.

The model was proposed by Black and Scholes in 1973, and gave theoretical support for trading options to hedge positions, a practice that was uncommon due to the lack of solid mathematical support. From this model, we are able to calculate the ideal price of an option using numerous availiabe factors. There are many different variations on this model, all of which seek to improve the base model in some way. These improvements, however, often come at the cost of a massive increase in mathematical complexity

This document will focus on the basic model, to serve as an educational jumping off point.

\end{abstract}

\section{Notation}
As in all financial models, there is some notation involved within the model.

\begin{itemize}
 \item[] C = Call option price 
 \item[] S = Current stock price
 \item[] K = Strike price of the option
 \item[] r = Risk-free interest rate
 \item[] $\sigma$ = Volatility
 \item[] t = Time to option maturity
 \item[] N = normal cumulative distribution
\end{itemize}

\newpage 

\section{Black-Scholes Equation}
The Black-Scholes equation describes the price of an option over time

\begin{equation}
    \huge
    \frac{\partial \mathrm C}{\partial \mathrm t}
    +
    \frac{1}{2}\sigma^{2}S^{2}
    \frac{\partial^{2} \mathrm C}{\partial \mathrm C^{2}}
    +
    rS\frac{\partial \mathrm C}{\partial \mathrm S}
    =
    rC
    \label{eq:(1)}
\end{equation}

Note that equation \ref{eq:(1)} is a partial differential equation. The solution to this equation gives us the Black-Scholes formula.

\section{Black-Scholes formula for European option price}

\subsection{Difference between European and American options markets}

\quad The primary difference, and the one that affects the Black-Scholes model, is the date when an option is able to be exercised. In American markets, options can be exercised anytime before they expire, whereas European options can only be exercised on the expiration date.
\end{document}
